\chapter{関連研究}

\section{動物行動認識}

動物行動認識は,映像データから動物の振る舞いを自動的に解析する技術であり,
その応用範囲は基礎的な生態学研究から動物園における飼育管理まで多岐にわたる.
動物の行動は,個体の精神的,身体的,および認知的な状態を色濃く反映することが知られており \cite{broom2006},
近年では特に,アニマルウェルフェア(動物福祉)を客観的に評価するための重要な指標として注目されている \cite{13,14}.

中でも,常同行動と呼ばれる反復的な異常行動は,
不適切な飼育環境や慢性的なストレスに起因して発現することが報告されており,
動物の健康状態や福祉水準を評価する上で極めて重要な指標となる \cite{13}.
従来,これらの行動評価は専門家による長時間の目視観察や記録に依存してきた.
しかし,人手による観察は多大な労力と時間を要するだけでなく,
観察者間の主観的ばらつきや,
24時間にわたる連続的なモニタリングが困難であるといった課題を抱えている.
そのため,低コストかつ継続的な運用が可能であり,
定量的な行動データを自動的に取得できるシステムの開発が強く求められている.

こうした背景のもと,近年の深層学習技術の発展は,
動物行動解析の分野に大きな変革をもたらした.
Convolutional Neural Networks(CNN)や
Vision Transformer(ViT)の登場により,
画像分類や物体検出の精度は飛躍的に向上し,
特定のデータセットにおいては,
人間と同等,あるいはそれ以上の認識性能が報告されている.
また,DeepLabCut \cite{mathis2018deeplabcut} に代表される姿勢推定技術の進展により,
映像から動物の関節位置を高精度に推定し,
微細な動作を定量的に解析することも可能となった.

\section{実環境における動物行動解析の事例}
動物園環境における代表的な解析事例として,
Wang ら \cite{wang} は,
監視カメラ映像を用いた
シロクマの常同行動検出手法を提案している.
この研究では,
時間平均画像を用いた背景差分により動体を検出し,
その重心軌跡の周期性を解析することで,
往復歩行などの常同行動を特定している.
照明変化や天候の影響を受けやすい屋外環境において,
ロバストな検出を実現した点は,
実運用を想定した重要な成果である.

一方で,
同手法は特定の動物種(シロクマ)および
移動軌跡という限定的な特徴量に依存しており,
移動を伴わない行動や,
外観の異なる他種への適用には課題が残る.
より多様な行動を種横断的に認識するためには,
軌跡情報に加え,
身体の動きや外観の変化を含む
高次元な表現学習が必要である.

\section{表現学習と行動埋め込み}

行動認識におけるアノテーションコストを削減し,未知の行動に対応するためのアプローチとして,
「表現学習(Representation Learning)」および「距離学習(Metric Learning)」が注目されている.
表現学習は,意味的に類似したサンプル(同じ行動)が特徴空間上で近接し,
意味的に異なるサンプル(異なる行動)が分離されるような
判別的な埋め込み空間(Embedding Space)を構築することを目的とする [cite: 40].

FaceNet \cite{6} によって広く知られるようになったトリプレット損失(Triplet Loss)などの距離学習手法は,
クラス分類層を持たないため,学習データに含まれないクラスや,
サンプル数の少ない希少な行動に対しても,
特徴空間上の距離に基づいてクラスタリングや異常検知が可能となる.
動物行動認識の文脈においても,この埋め込みに基づくアプローチは,
教師なしあるいは半教師ありの行動解析を実現するための重要な基盤となる [cite: 42].


\section{敵対的学習による不変表現学習}

高次元な特徴表現を学習する際,
行動認識に不要な属性情報
(動物種や背景など)が特徴量に強く残存すると,
汎化性能を阻害する要因となる.
この問題に対処する手法として,
敵対的学習(Adversarial Learning)を用いたドメイン適応が提案されている.

Domain-Adversarial Neural Networks(DANN)\cite{45,234} では,
特徴抽出器の後段に
タスク分類器とドメイン識別器を配置し,
勾配反転層を用いて学習を行う.
ドメイン識別器がドメインを識別できないように
特徴抽出器を更新することで,
特徴空間からドメイン特有の情報を抑制し,
タスクに本質的な情報のみを保持した
不変表現の獲得が可能となる.
人物行動認識などの分野では,
この枠組みの有効性が広く示されているが 
動物行動認識において,
動物種をドメインとして扱い,
種に依存しない表現を獲得する試みは
未だ十分に検討されていない
