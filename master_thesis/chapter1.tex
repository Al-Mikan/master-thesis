\chapter{序論}

\section{研究背景}
近年,動物行動の定量的な解析は,動物園における個体管理や行動モニタリング,
さらには動物福祉の向上を目的として重要性を増している.
従来,動物行動の評価は飼育員や研究者による目視観察に基づいて行われてきたが,
長時間にわたる継続的な観察には多大な人手と時間を要するという課題がある.
このような背景から,映像データを用いたコンピュータビジョン技術による
動物行動認識の自動化が強く望まれている.

深層学習に基づく行動認識技術は,
特定のデータセットにおいて高い認識性能を達成している.
しかし,これらの手法の多くは膨大な行動ラベル付きデータを必要とするため,
新たな動物種や撮影環境へ適用する際には,
再度大規模なアノテーション作業が必要となる.
特に動物園環境では,動物種ごとの外観差や行動様式の多様性に加え,
照明条件や檻による遮蔽といった撮影環境の影響が大きく,
特定の環境で学習されたモデルを
そのまま他環境へ適用(汎化)することが困難であるという問題がある.

このような課題に対し,
特定のドメインに依存しない行動に関する本質的な特徴を抽出し,
動物種や環境の変化に頑健な表現を獲得することが重要である.
その有効なアプローチとして,
映像から得られる外観情報に加え,
オプティカルフローに基づく運動情報を併用する
マルチモーダル表現学習が注目されている.
静的な外観特徴と動的な運動特徴を統合することで,
単一のモダリティでは捉えきれない
複雑な行動パターンを表現可能になると期待される.

さらに,表現学習の過程において,
行動認識に不要な属性情報を抑制する手法として,
敵対的学習を用いた不変表現学習
(Invariant Representation Learning)が提案されている.
これらの手法は,人や物体の認識分野において,
ドメイン間の差異を克服した特徴表現の獲得に成功しており,
動物行動認識への応用も期待されている.
しかし,マルチモーダル表現学習と敵対的学習を組み合わせ,
動物種および環境の多様性を同時に克服することを目的とした研究は,
未だ十分に検討されていない.

\section{研究目的}
本研究の目的は,
動物種および撮影環境の多様性に頑健な動物行動認識を実現するため,
マルチモーダル表現学習に基づく共有表現空間の構築手法を提案することである.
具体的には,RGB 映像から得られる外観情報と,
オプティカルフローに基づく運動情報を統合することで,
行動に関する判別的な特徴表現を学習する.

本研究の核心的な提案として,
獲得される表現から動物種固有のバイアスを排除することを目的とした,
種分類器との敵対的学習を導入する.
これにより,種の違いというドメインの壁を越え,
未知の動物種や環境に対しても高い汎化性能を持つ,
汎用的な行動表現の獲得を目指す.

本研究では,
大規模動物行動データセットを用いて提案手法の学習を行い,
その有効性を検証する.
さらに,未知環境で撮影された動物園のシロクマ映像を用いた評価実験を通じ,
実環境における適用可能性と
提案手法の頑健性を明らかにする.

\section{本論文の構成}

本論文の構成を以下に示す.
第2章では,動物行動認識およびマルチモーダル表現学習,
ならびに敵対的学習に関する関連研究について述べる.
第3章では,本研究で提案する
マルチモーダル表現学習および敵対的学習に基づく手法の詳細を説明する.
第4章では,実験に用いたデータセットや実装条件,
評価指標について述べる.
第5章では,提案手法の実験結果を示し,
単一モダリティ手法との比較や分析を行う.
第6章では,実験結果に基づく考察を行い,
提案手法の有効性や限界について議論する.
最後に,第7章では本研究のまとめと今後の課題について述べる.